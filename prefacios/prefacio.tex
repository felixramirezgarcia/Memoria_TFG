\begin{center}
{\large\bfseries InfoJaus: Aplicación Web de consultas inmobiliarias en Andalucía}\\
\end{center}

\begin{center}
Félix Ramírez García\\
\end{center}

\begin{flushleft}
	\noindent{\textbf{Palabras clave}: APP web, REST API, Base de datos NoSQL, información inmobiliaria ......}\\
	
	\vspace{0.7cm}
	\noindent{\textbf{Resumen\vspace{0.5cm}}}\\
	La dinamización actual del sector inmobiliario muestra la necesidad de una búsqueda de información de calidad a través de las diferentes plataformas que dan éste soporte.  
	El objetivo de este proyecto es dar el servicio de búsqueda de información inmobiliaria en Andalucía con un modelo de negocio realista y escalable, en el que se generan beneficios por suscripción a la aplicación web.
	Para la obtención de la información inmobiliaria ha sido necesario el desarrollo de otra aplicación que se ejecuta periódicamente para mantener actualizada la base de datos mediante peticiones a la API de la plataforma Nestoria.
	La información se visualiza realizando consultas mediante cuadros de mando territoriales en la plataforma web alojada en Heroku, plataforma que realiza consultas a una base de datos NoSQL .
\end{flushleft}

\newpage %inserta un salto de página


